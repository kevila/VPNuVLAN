\chapter{Einleitung}



Die Kommunikation über das Internet, die Verfügbarkeit von digitalen Medien nahezu jederzeit und überall und das Anhäufen von digitalen Daten  sind  heute Teil des Alltags, sowohl im privten wie auch im geschäftlichen Bereich. Auch für Unternehmen  sind das Netzwerk und das Internet ein wichtiger wirtschaftlicher Faktor geworden \cite{lipp2007vpn}. Dabei sind  Flexibilität und Sicherheit von großer Bedeutung, allerdings spielen gerade in einem Unternehmen auch die Kosten eine wichtige Rolle. Durch virtuelle Netzwerkstrukturen wie virtuelles LAN (VLAN) und virtuelle Private Netze (VPN) ist es möglich, kostengünstige Lösungen zu schaffen. 

Dem Aspekt der Sicherheit muss dabei besondere Beachtung geschenkt werden, da Angriffe auf Firmennetze erheblichen finanziellen Schaden verursachen können.\\  

Eine Studie des WIK (Wissenschaftliches Institut für Infrastruktur und Kommunikationsdienste) \cite{wik2017KMU} ergab, dass zwar 64 \% der kleinen Unternehmen\footnote{Unternehmen mit bis zu 49 Mitarbeitern werden hier als kleine Unternehmen bezeichnet.} in Deutschland angeben, IT-Sicherheit habe eine hohe Bedeutung, eine Sicherheitsanalyse aber nur von $20\; \%$ durchgeführt wurde.
Dabei benutzen 94 \% der kleinen Unternehmen laut dieser Umfrage PC-Arbeitsplätze mit Internetzugang und in  75 \%  kommen mobile Endgeräte wie Smartphones, Notebooks und Tablets zum Einsatz. Gleichzeitig geben nur 31 \% der kleinen Unternehmen an, VPN zu nutzen. Bereits Virenangriffen ausgesetzt waren 53 \% der Unternehmen. 





  %Im Jahr 2016 gaben $50\%$ der kleinen Unternehmen an schon mal einen Cyberangriff festgestellt zu haben. Der Prozetnsatz wächst mit der Unternehmensgröße weiter an\footnote{Cybersicherheits-Umfrage der Allianz für Cybersicherheit 2016}. 




Im Folgenden sollen die virtuellen Netzwerkstrukturen VPN und VLAN vorgestellt werden. In Kapitel 2 soll zunächst kurz ein Überblick über des Aufbau eines sicheren Unternehmensnetzwerks gegeben werden. Kapitel 3 gibt eine Einführung in das Thema VLAN und der damit verbundenen logischen Segmentierung des lokalen Netzes eines Unternehmens. Im vierten Kapitel wird daraufhin die virtuelle Standorterweiterung eines lokalen Unternehmensnetzes, bzw. der Zugriff aus der Ferne durch VPN vorgestellt. 
Schließlich bietet Kapitel 5 eine kurze Zusammenfassung und einen Ausblick. 































%Die Digitalisierung bringt sowohl für Privathaushalte, als auch für Unternehmen viele neue Möglichkeiten. Durch die zunehmende Vernetzung ergeben sich aber auch vielfältige Angriffsmöglichkeiten, die die Sicherheit von Privathaushalten und Unternehmen bedrohen. IT-Sicherheit wird auch in kleinen Unternehmen zunehmend wichtiger, die Umsetzung dieser fällt jedoch gerade kleinen Unternehmen häufig noch schwer. Eine Studie des WIK (Wissenschaftliches Institut für Infrastruktur und Kommunikationsdienste} ergab, dass zwar $64\%$ der kleinen Unternehmen in Deutschland angeben IT-Sicherheit habe eine hohe Bedeutung, eine Sicherheitsanalyse aber nur von $20\%$ durchgeführt wurde \cite{wik2017KMU}.   


%....%
%Das Netzwerk und die Internetanbindung wurden in den letzten Jahren zu einem wichtigen Faktor, auch für den wirtschaftlichen Erfolg eines Unternehmens \cite{lipp2007vpn}.

