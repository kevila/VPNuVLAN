\chapter{Zusammenfassung und Ausblick}

Aus Kostengründen und wegen der großen Flexibilität setzen gerade kleine Unternehmen gerne auf virtuelle Netzwerklösungen. Mit virtuelle LAN lässt sich das lokale Netzwerk der Firma segmentieren und so eine zusätzliche Hürde für Angreifer schaffen.

Virtuelle private Netzwerke bieten eine Möglichkeit aus der Ferne mit einem Hohen Maß an Sicherheit auf das Firmennetzwerk zuzugreifen. 

Die Sicherheit der virtuellen Infrastrukturen hängt von Konfiguration, Wartung und Administration ab, Unwissenheit oder Nachlässigkeit können zum Sicherheitsrisiko werden. \\   

Ein Anstatz für die Verbesserung der IT-Sicherheit ist \emph{Security by Design}. Am Beispiel eines Netzwerkes sollten Vernetzung und Sicherheitsmaßnahmen nicht getrennt voneinander betrachtet werden, sondern ein Netzwerk aus intrisisch sicheren Komponenten Aufgebaut werden \cite{nicholson2018blurring}. Auch in dem hier häufig zitierten Buch \emph{VPN Virtuelle Private Netze, Aufbau und Sicherheit} von Manfred Lipp aus dem Jahr 2007, ist davon die Rede wie der Netzwerker und der Sicherheitsverantwortliche gegensätzliche Ziele verfolgen: Transparente Kommunikation vs. eingeschränkte Kommunikation, und auch Lipp betont schon, dass diese Trennung langsam aufgehoben wird und eine Zusammenarbeit nötig ist \cite{lipp2007vpn}. 

Am Beispiel von VLAN wird deutlich, wie wichtig es ist bei dem Design einer Technologie die Sicherheit von Anfang an mit zu betrachten. VLAN wurde eben ohne dieses Designprinzip entwickelt und so lassen sich zum Beispiel die hier Verwendeten Protokolle leicht durch Angriffe wie double Tagging Missbrauchen. Sicherheitsmechanismen sollten nicht nur nachträglich konfiguriert werden, sondern Teil der Protokolle sein. In diesem Fall wäre das ein Automatisches erkennen von doppelten Tags. Auch das Protokoll zum Automatischen erstellen von Trunkverbindungen zwischen Switchports, ist für den Anwender sehr bequem, aber für einen Angreifer eben auch. 

Aber auch wenn Netzwerke der Zukunft aus Komponenten bestehen werden, die unter dem Aspekt der Sicherheit entwickelt werden, %FORMULIERUNG
wird es voraussichtlich immer Schwachstellen und Angriffspunkte geben. Allein schon die Menschliche Komponente bleicht unberechenbar. Ein gutes Sicherheitskonzept und Mitarbeiterschulung und Aufklärung helfen, um die Risiken zu Minimieren.