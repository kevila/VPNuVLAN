\chapter{Virtuelle LANs}

Virtuelle Netze oder auch virtuelle local area networks (VLAN) werden genutzt, um bestehende physische LANs in logische Abschnitte zu Unterteilen, oder auch um mehrere physische Netzabschnitte virtuell zu verbinden. VLANs werden mithilfe von Switches gebildet. Switche sind Netzwerkkomponenten, welche die Schichten eins und zwei des Internetprotokollstapels (vgl. Anhang \ref{A1}) implementieren, sie bilden heute die Zentralen Netzknoten in (lokalen) Netzwerken \cite{zisler2018computer}. 

%nutzen hinzufügen

\subsection{Funktionsweise}

% portbasiertes und tagged Vlan

Es gibt zwei Arten von Vlans, das Portbasierte und das Paketbasierte VLAN. 

